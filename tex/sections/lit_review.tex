\section{Literature Review}

Agent-Based Modeling (ABM) has established itself as a critical instrument in financial economics, offering explanations for "stylized facts" that traditional equilibrium models often fail to reproduce.

\subsection{Agent-Based Models in Finance}
Seminal works in the field have laid the groundwork for understanding market complexity. Lux and Marchesi (1999) demonstrated that the interaction between fundamental traders and chartists can generate "fat tails" in return distributions and volatility clustering, phenomena that are ubiquitous in real-world markets. Cont (2001) further systematized these stylized facts, providing a robust benchmark for the validation of Agent-Based Models.

\subsection{Information Asymmetry and Sentiment}
The influence of news sentiment on market dynamics has been the subject of extensive research. Tetlock (2007) provided compelling evidence of the link between media pessimism and market downturns. In the context of ABM, studies such as LeBaron (2006) often model the heterogeneity of beliefs. However, few studies have explicitly separated the information source into "headline" and "content," a distinction that constitutes the central focus of our research.

\subsection{Volatility and Market Making}
The role of market makers in providing liquidity and their impact on volatility was investigated by Chiarella et al. (2009). Their work highlights how Delta-hedging strategies can create feedback loops that affect price dynamics, particularly during periods of market stress. Our model extends this approach by incorporating an options market with dynamic strikes, allowing for a more granular analysis of volatility surfaces.

\textbf{Key Sources}
The theoretical foundation of this research relies on several key texts. LeBaron (2006) provides a comprehensive overview of agent-based computational finance. Lux and Marchesi (1999) offer insights into scaling and criticality in stochastic multi-agent models. Cont (2001) discusses the empirical properties of asset returns, essential for model validation. Finally, Tetlock (2007) explores the role of media in stock market pricing, directly relevant to our sentiment analysis component.
