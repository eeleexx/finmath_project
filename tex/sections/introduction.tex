\section{Introduction}

\subsection{Motivation and Relevance}
Financial markets are characterized by a high degree of information asymmetry, a phenomenon that has intensified in the age of digital media. While news spreads instantly across global networks, the quality of its perception by market participants varies significantly. Retail investors, often referred to as "Noise Traders," frequently react to emotionally charged headlines or clickbait, making impulsive decisions based on surface-level information. In contrast, professional participants, or "Sophisticated Traders," delve deeper, analyzing the nuanced content of news articles to form a more comprehensive view. This disparity in information processing can lead to substantial distortions in asset pricing, particularly within the derivatives market where volatility plays a central role.

Agent-Based Modeling (ABM) has emerged as a powerful tool for investigating these microstructural effects. By creating "laboratory" conditions that are unavailable when analyzing only historical data, ABM allows researchers to simulate the interactions between heterogeneous agents and observe the emergent market dynamics.

\subsection{Project Goal}
The primary objective of this project is to develop a multi-agent simulation to investigate the impact of information asymmetry on the options market. We aim to understand how the interaction between traders reacting to headlines and traders analyzing content affects market dynamics. Specifically, we examine the formation of the "Volatility Smile," the spread between Implied (IV) and Realized (RV) volatility, and the profitability and risk exposure of Market Makers.